\documentclass[12pt,a4paper]{article}

%% ─── Packages ────────────────────────────────────────────────────────────────
\usepackage[utf8]{inputenc}
\usepackage[T1]{fontenc}
\usepackage[margin=2.5cm]{geometry}
\usepackage{graphicx}
\usepackage{subcaption}
\usepackage{booktabs}
\usepackage{hyperref}
\usepackage{xcolor}
\usepackage{parskip}
\usepackage{enumitem}
\usepackage{float}
\usepackage{amsmath}
\usepackage{microtype}
\usepackage{caption}
\usepackage{array}
\usepackage{multirow}
\usepackage{longtable}
\captionsetup{font=small, labelfont=bf, justification=justified}
\hypersetup{colorlinks=true, linkcolor=blue, citecolor=blue, urlcolor=blue}

\begin{document}

%% ═══════════════════════════════════════════════════════════════════════════════
%%  TITLE PAGE
%% ═══════════════════════════════════════════════════════════════════════════════
\begin{titlepage}
  \centering
  \vspace*{2cm}
  {\LARGE\bfseries A Geographical and Economic Analysis\\[0.4em]
   of the New York City Airbnb Market\\[0.4em]
   --- Final Visualisation Report ---\par}
  \vspace{1.5cm}
  {\large Visualisation Project --- Milestone 4 Final Report\par}
  \vspace{1cm}
  {\large Hüseyin Soykök\quad|\quad Student ID: 3804308\quad|\quad Group 24\par}
  \vspace{0.5cm}
  {\large February 2026\par}
  \vspace{2cm}
  \begin{abstract}
    \noindent
    This report documents the design, implementation, and evaluation of an interactive
    multi-perspective dashboard for the New York City Airbnb Open Data (2019) dataset
    (47,924 listings, 16 attributes). The dashboard is built in Python with
    PySide6 and Plotly, and provides five persona-driven views: \emph{Traveler},
    \emph{Investor}, \emph{Journalist}, \emph{Competitor}, and \emph{Regulator}.
    Each view encodes a specific analytical task through carefully chosen visual
    idioms --- geographical symbol maps, violin plots, bar charts, histograms, and
    scatter plots --- all linked through a shared filter panel. The system supports
    three rendering themes (Dark, Light, Grayscale) to address accessibility needs.
    This report describes the domain situation, data abstraction, visual encodings,
    implemented interactions, design rationale, and key findings derived from each
    perspective.
  \end{abstract}
\end{titlepage}

\tableofcontents
\newpage

%% ═══════════════════════════════════════════════════════════════════════════════
%%  1. INTRODUCTION
%% ═══════════════════════════════════════════════════════════════════════════════
\section{Introduction}

The short-term rental market in New York City represents one of the world's most
competitive and geographically complex accommodation ecosystems. The NYC Airbnb
Open Data (2019) dataset, published on Kaggle under CC0 1.0 (Public Domain), contains
48,895 rows and 16 attributes capturing listing location, price, room type, host
information, review counts, and availability, constituting a rich substrate for
multi-perspective data visualisation analysis \cite{dgomonov2019}.

The overarching goal of this project is to transform raw tabular data into an
interactive, persona-driven dashboard that allows five distinct user archetypes ---
a budget traveller, a property investor, a data journalist, a hotel competitor,
and a city regulator --- to answer their specific analytical questions through
guided exploration. Following the Munzner What--Why--How framework \cite{munzner2014},
each view is designed to match its user's task abstraction (discover, compare,
identify, locate) with an appropriate visual idiom (symbol map, violin plot,
bar chart, histogram) \cite{brehmer2013}.

The implemented system consists of a PySide6 desktop application \cite{pyside6}
with Plotly-generated interactive figures \cite{plotly2015} embedded in
\texttt{QWebEngineView} widgets. A shared filter panel propagates user-selected
constraints (borough, room type, price range, minimum reviews, maximum nights)
to all active views simultaneously, enabling coordinated multi-view exploration
\cite{heer2012}.

%% ═══════════════════════════════════════════════════════════════════════════════
%%  2. DATASET
%% ═══════════════════════════════════════════════════════════════════════════════
\section{Dataset Description}

\subsection{Source and License}
The dataset is the \emph{New York City Airbnb Open Data (2019)}, sourced from
\url{https://www.kaggle.com/datasets/dgomonov/new-york-city-airbnb-open-data},
published by user \texttt{dgomonov} \cite{dgomonov2019}. The license is CC0 1.0
Public Domain Dedication, permitting free academic and commercial use without
restriction.

\subsection{Schema and Attribute Roles}

Table~\ref{tab:schema} summarises the 16 attributes, their semantic types,
measurement units, and value ranges.

\begin{table}[H]
  \centering
  \caption{Dataset schema: attribute roles, types, units, and value ranges.}
  \label{tab:schema}
  \small
  \begin{tabular}{llllp{4.5cm}}
    \toprule
    \textbf{Attribute} & \textbf{Role} & \textbf{Type} & \textbf{Unit} & \textbf{Range / Domain} \\
    \midrule
    \texttt{id}                          & Key       & Quantitative   & Unique ID  & 48,895 unique integers \\
    \texttt{name}                        & Attribute & Categorical     & Text       & $\sim$47,900 unique strings \\
    \texttt{host\_id}                    & Key       & Quantitative   & Unique ID  & 37,457 unique hosts \\
    \texttt{host\_name}                  & Attribute & Categorical     & Text       & $\sim$11,400 unique strings \\
    \texttt{neighbourhood\_group}        & Attribute & Categorical (Spatial) & ---  & 5 boroughs \\
    \texttt{neighbourhood}               & Attribute & Categorical (Spatial) & ---  & 221 neighbourhoods \\
    \texttt{latitude}                    & Attribute & Quantitative (Spatial) & Degree & [40.499, 40.913] \\
    \texttt{longitude}                   & Attribute & Quantitative (Spatial) & Degree & [$-$74.244, $-$73.712] \\
    \texttt{room\_type}                  & Attribute & Categorical     & ---        & 3 categories \\
    \texttt{price}                       & Attribute & Quantitative   & USD        & [0, 10,000] (contains outliers) \\
    \texttt{minimum\_nights}             & Attribute & Quantitative   & Nights     & [1, 1,250] (contains outliers) \\
    \texttt{number\_of\_reviews}         & Attribute & Quantitative   & Count      & [0, 629] \\
    \texttt{last\_review}                & Attribute & Temporal       & Date       & 2011-03-28 to 2019-07-08 \\
    \texttt{reviews\_per\_month}         & Attribute & Quantitative   & Count/Month & [0.01, 58.5] \\
    \texttt{calculated\_host\_listings\_count} & Attribute & Quantitative & Count & [1, 327] \\
    \texttt{availability\_365}           & Attribute & Quantitative   & Days       & [0, 365] \\
    \bottomrule
  \end{tabular}
\end{table}

\subsection{Data Quality and Preprocessing}

The dataset presents three notable quality challenges. First, \textbf{extreme outliers}
in \texttt{price} (\$0 and up to \$10,000) and \texttt{minimum\_nights} (up to
1,250 nights) distort statistical summaries and break visualisation scales. All price
visualisations are therefore filtered to \$10--\$1,000 prior to rendering, and
minimum-nights analysis is capped at 30 nights unless otherwise stated. Second,
approximately 10,052 rows ($\approx$20\%) have null values in \texttt{last\_review}
and \texttt{reviews\_per\_month}; these are not random errors but are structurally
tied to listings with zero reviews and are treated as \emph{inapplicable} rather
than \emph{missing} data, excluded only from review-based calculations. Third, the
dataset is a \textbf{static 2019 snapshot} \cite{dgomonov2019}, predating the
COVID-19 pandemic and the subsequent structural shift in NYC's rental market
\cite{zervas2017}; all findings must be interpreted within this temporal context.

%% ═══════════════════════════════════════════════════════════════════════════════
%%  3. SYSTEM ARCHITECTURE
%% ═══════════════════════════════════════════════════════════════════════════════
\section{System Architecture and Interaction Design}

\subsection{Technology Stack}
The dashboard is implemented as a \textbf{PySide6} desktop application \cite{pyside6}.
Interactive charts are generated with \textbf{Plotly} (Express and Graph Objects
APIs) \cite{plotly2015} and rendered inside \texttt{QWebEngineView} widgets.
Data wrangling is performed with \textbf{Pandas} \cite{pandas}. The application
follows an MVC-like architecture: a central \texttt{FilterState} object holds the
current filter selections; each view registers as a listener and re-renders its
charts whenever the filter state changes.

\subsection{Filter Panel}

\begin{figure}[H]
  \centering
  \includegraphics[width=0.65\textwidth]{figures/filter_panel.jpg}
  \caption{The shared Filter Panel, displayed as a collapsible sidebar.
    The panel provides six interactive controls: (1) Borough checkboxes for
    spatial scoping by \texttt{neighbourhood\_group}; (2) a Neighbourhood dropdown
    for fine-grained filtering to a single \texttt{neighbourhood}; (3) Room Type
    checkboxes (\emph{Entire home/apt}, \emph{Private room}, \emph{Shared room});
    (4) dual Price Range sliders (\$100--\$150 shown, constraining \texttt{price});
    (5) a Minimum Reviews slider (here $\geq 0$); and (6) a Max Stay Requirement
    slider (here $\leq 30$ nights, constraining \texttt{minimum\_nights}).
    The ``Active:'' badge strip at the top reflects the current filter state as
    readable tokens. All six controls propagate simultaneously to every open
    perspective view, implementing a coordinated brushing-and-linking interaction
    pattern \cite{heer2012} without any additional user action.}
  \label{fig:filter_panel}
\end{figure}

\subsection{Theme System}

The application supports three rendering themes to address varying use-case and
accessibility requirements. Figure~\ref{fig:themes} illustrates all three modes
on the Traveler view.

\begin{figure}[H]
  \centering
  \begin{subfigure}[t]{0.48\textwidth}
    \includegraphics[width=\textwidth]{figures/theme_dark.jpg}
    \caption{Dark Mode. Background \texttt{\#0d1117}, foreground text
      \texttt{\#e6edf3}. All Plotly chart papers and plot backgrounds are
      overridden to match the dark theme. Recommended for extended use in
      low-light environments.}
    \label{fig:theme_dark}
  \end{subfigure}\hfill
  \begin{subfigure}[t]{0.48\textwidth}
    \includegraphics[width=\textwidth]{figures/them_light.jpg}
    \caption{Light Mode. White background (\texttt{\#ffffff}), dark text
      (\texttt{\#1f2328}). Chart backgrounds switch to white; all axis labels
      and grid lines adapt accordingly. Optimised for daytime use and
      presentation/printing contexts.}
    \label{fig:theme_light}
  \end{subfigure}
  \vspace{0.5em}
  \begin{subfigure}[t]{0.48\textwidth}
    \includegraphics[width=\textwidth]{figures/traveller_grayscale.jpg}
    \caption{Grayscale Mode. All hue-based colours are desaturated to a
      luminance-ranked grayscale palette \cite{moreland2009}. This mode
      guarantees readability for the approximately 8\% of male users with
      colour-vision deficiency \cite{wcag21}. Stat cards retain colour even
      in this mode. Charts rely on position, length, and pattern encoding
      rather than hue, so no information is lost.}
    \label{fig:theme_grayscale}
  \end{subfigure}
  \caption{Three rendering themes of the dashboard (Traveler View shown). The
    active theme is toggled via the sidebar's \emph{Grayscale} and \emph{Dark Mode}
    switches.}
  \label{fig:themes}
\end{figure}

\subsection{Application Overview}

\begin{figure}[H]
  \centering
  \includegraphics[width=\textwidth]{figures/app_overview.jpg}
  \caption{Full application overview showing the Traveler View in Dark Mode with
    the Brooklyn and Queens boroughs selected and a \$100--\$150 price range active.
    The left sidebar contains the perspective switcher (Traveler, Investor,
    Regulator, Competitor, Journalist) and tool section. The top-right status bar
    displays real-time aggregate statistics: 472 listings, average price \$125,
    445 hosts. The main canvas renders the value-focused map in the top half and
    the price histogram alongside the borough bar chart in the lower half, all
    updating instantaneously when any filter slider or checkbox is adjusted.}
  \label{fig:app_overview}
\end{figure}

%% ═══════════════════════════════════════════════════════════════════════════════
%%  4. TRAVELER VIEW
%% ═══════════════════════════════════════════════════════════════════════════════
\section{Traveler View: Finding Value-for-Money Accommodation}

\subsection{Persona and Task}

The Traveler perspective is designed for \textbf{Hüseyin}, a budget-conscious tourist
planning a 5-day visit to NYC. His budget is firm at \$100--\$150 per night; he
requires a minimum of 100 reviews as a proxy for listing quality and safety; and
he prefers \emph{Entire home/apt} listings. He is unfamiliar with NYC geography
and therefore needs a spatial overview before he can select concrete options.

His primary task is a \textbf{Discover / Locate} task \cite{brehmer2013}: given
strict filter criteria (price, reviews, room type), find the geographic clusters
of ``value gem'' listings. Secondary tasks include \textbf{Compare} (which borough
offers the best value score) and \textbf{Identify} (specific listings worth
booking).

\subsection{Value-Focused Listing Map}

\begin{figure}[H]
  \centering
  \includegraphics[width=\textwidth]{figures/traveler_map.jpg}
  \caption{Value-Focused Listing Map (Scatter Mapbox, \texttt{px.scatter\_mapbox}).
    Each dot represents a single Airbnb listing. Two visual channels encode the
    two dimensions that matter most to the traveller simultaneously:
    \textbf{colour} encodes \texttt{price} via a sequential green--yellow--red
    diverging scale (green = \$100 [cheapest], red = \$150 [upper budget limit]);
    \textbf{dot size} encodes \texttt{number\_of\_reviews} (larger dot = more
    reviewed = more popular). This dual encoding means that the ``ideal'' listing
    from Hüseyin's perspective is a \emph{large green dot} --- affordable and
    highly vetted \cite{ware2004}. The map is shown here with Brooklyn and Queens
    selected, a \$100--\$150 price range, and $\geq$100 reviews active: 472
    qualifying listings remain. A hover tooltip surfaces the listing name, price,
    room type, neighbourhood, availability, and minimum nights. Geometric zoom
    and pan allow street-level inspection. The colour scale legend is fixed at
    the right to ensure stable colour--price mapping across filter changes.}
  \label{fig:traveler_map}
\end{figure}

\subsection{Price Distribution Histogram}

\begin{figure}[H]
  \centering
  \includegraphics[width=0.75\textwidth]{figures/traveler_price_histogram.jpg}
  \caption{Price Distribution Histogram (\texttt{px.histogram}, bin width \$5,
    capped at $\leq$\$500). The x-axis shows \texttt{price} per night in USD;
    the y-axis shows absolute listing count. Two vertical reference lines are
    superimposed: a \textbf{green dashed} line for the median (\$125) and an
    \textbf{orange dashed} line for the mean (\$125), which here coincide due
    to the symmetric price distribution enforced by the active filters. The
    histogram reveals a bimodal pattern within the \$100--\$155 range: a dominant
    peak at \$100 (2,080 listings), a secondary concentration around
    \$120--\$125 (1,165 listings), and a notable spike at \$150 (2,020 listings)
    attributable to hosts pricing at the round-number ceiling \cite{tufte1983}.
    Bins between \$130 and \$145 show lower counts (300--640 listings), confirming
    that Hüseyin's budget range is realistic and supply-rich at both the lower
    and upper ends.}
  \label{fig:traveler_histogram}
\end{figure}

\subsection{Average Price by Borough Bar Chart}

\begin{figure}[H]
  \centering
  \includegraphics[width=0.75\textwidth]{figures/traveler_brough_bar.jpg}
  \caption{Average Price by Borough (vertical bar chart, \texttt{px.bar}).
    Each of the five boroughs is represented by a distinct bar, colour-coded
    categorically (Bronx: blue; Brooklyn: red; Manhattan: orange; Queens: green;
    Staten Island: purple). The y-axis encodes average price per night (USD),
    with bars annotated with exact dollar values. Manhattan leads at \$126/night,
    followed by Brooklyn (\$124), Queens (\$122), Bronx (\$121), and Staten
    Island (\$119). The inter-borough price spread is only \$7, indicating that
    once Hüseyin applies his budget constraints, all boroughs offer comparable
    average prices. The length channel is the primary encoding, consistent with
    Cleveland and McGill's \cite{cleveland1984} finding that position along a
    common scale is the most accurate preattentive channel for magnitude judgment.}
  \label{fig:traveler_borough_bar}
\end{figure}

\subsection{Room Type Distribution Bar Chart}

\begin{figure}[H]
  \centering
  \includegraphics[width=0.75\textwidth]{figures/traveler_roomtype_bar.jpg}
  \caption{Room Type Distribution Bar Chart (\texttt{px.bar}).
    The x-axis lists the three room types (\emph{Entire home/apt},
    \emph{Private room}, \emph{Shared room}); the y-axis shows absolute
    listing count. Each bar is colour-coded by room type (blue: Entire
    home/apt; green: Private room; orange: Shared room) and annotated with
    both the absolute count and the percentage of the total.
    \emph{Entire home/apt} dominates with 8,164 listings (68.8\%),
    \emph{Private room} accounts for 3,620 (30.5\%), and \emph{Shared room}
    is a marginal category with only 85 listings (0.7\%). A bar chart is
    used here in preference to a pie chart because length-based comparison
    of counts is perceptually more precise than angle- or area-based judgment
    required by pie slices \cite{cleveland1984,munzner2014}. The near-zero
    count of Shared room is made unambiguous by the explicit axis scale,
    which a thin pie slice would under-represent.}
  \label{fig:traveler_roomtype_bar}
\end{figure}

\subsection{Value Score by Borough Bar Chart}

\begin{figure}[H]
  \centering
  \includegraphics[width=0.75\textwidth]{figures/traveler_value_bar.jpg}
  \caption{Value Score by Borough (horizontal bar chart with sequential colour
    scale, \texttt{px.bar}). The value score is computed as
    $\text{Value Score} = \texttt{number\_of\_reviews} / \texttt{price}$,
    representing reviews per dollar --- a composite proxy for price-normalised
    popularity \cite{munzner2014}. A sequential green colour scale (dark green
    = highest score) reinforces the magnitude ordering already encoded in bar
    length, providing a redundant visual channel \cite{ware2004} that makes
    the ranking immediately legible even in grayscale. Brooklyn achieves the
    highest value score ($\approx$0.22), marginally ahead of Bronx
    ($\approx$0.215) and Queens ($\approx$0.205), while Manhattan scores
    lowest ($\approx$0.185) despite its high number of listings.}
  \label{fig:traveler_value_bar}
\end{figure}

\subsection{Top 10 Best Value Listings Table}

\begin{figure}[H]
  \centering
  \includegraphics[width=\textwidth]{figures/traveler_table.jpg}
  \caption{Top 10 Best Value Listings (interactive QTableWidget, ranked by
    descending value score). Each row represents a single listing with seven
    columns: Rank, Listing Name, Neighbourhood, Price, Reviews, Minimum Nights,
    Room Type, and Value Score. Price cells are colour-coded green for budget
    entries and orange/red for higher-priced options. Minimum Nights cells are
    colour-coded green ($\leq$3 nights) and red (long minimum stay requirements).
    The top-ranked listing achieves a value score of 0.7 at \$399/night with
    287 reviews and a 2-night minimum. The table provides the most actionable
    output in the Traveler view, transitioning from spatial and statistical
    exploration to a concrete shortlist of bookable listings.}
  \label{fig:traveler_table}
\end{figure}

%% ═══════════════════════════════════════════════════════════════════════════════
%%  5. INVESTOR VIEW
%% ═══════════════════════════════════════════════════════════════════════════════
\section{Investor View: Identifying High-Yield Investment Opportunities}

\subsection{Persona and Task}

The Investor perspective is designed for \textbf{David}, a local resident
considering purchasing a property to list on Airbnb with the goal of maximising
return on investment. David is not interested in individual listings; he needs
aggregated market statistics to compare segments \cite{zervas2017}. His primary
task is \textbf{Discover / Compare} \cite{brehmer2013}: compare median prices,
revenue potential, and market saturation across boroughs and room types, and
identify the highest-yield segment.

\subsection{Estimated Annual Revenue Map}

\begin{figure}[H]
  \centering
  \includegraphics[width=\textwidth]{figures/investor_map.jpg}
  \caption{Estimated Annual Revenue Distribution (Scatter Mapbox,
    \texttt{px.scatter\_mapbox}). Each dot encodes a single listing, with
    \textbf{colour} representing estimated annual revenue (sequential
    blue--purple--orange--red scale: blue = \$0, red = \$50,000+) and
    \textbf{dot size} proportional to the same revenue measure, creating
    redundant encoding \cite{ware2004} for pre-attentive pop-out. Estimated
    revenue is computed as $\hat{R} = \texttt{price} \times
    \texttt{availability\_365}$. The map is shown zoomed into Midtown Manhattan,
    where the highest-revenue listings concentrate as large orange-to-red dots.
    The hover tooltip reveals the full data record including estimated revenue,
    price, and availability.}
  \label{fig:investor_map}
\end{figure}

\subsection{Price Distribution by Borough: Violin + Box Plot}

\begin{figure}[H]
  \centering
  \includegraphics[width=\textwidth]{figures/investor_violin_borough.jpg}
  \caption{Price Distribution by Borough --- Violin + Box composite
    (\texttt{go.Violin} with \texttt{box\_visible=True}). Each violin
    encodes the full kernel density estimate (KDE) of \texttt{price} for
    one borough, with a boxplot inset showing the five-number summary.
    Width at any price level is proportional to the density of listings
    at that price. All five boroughs exhibit broadly symmetric, unimodal
    distributions centred near \$125. Manhattan's violin shows a slight
    waist around \$115 indicating bimodality; Bronx and Staten Island extend
    to lower price floors ($\approx$\$87 and \$82 respectively). Unlike a
    bar chart that shows only a single average, the violin exposes whether
    a market is \emph{stable} (compact shape) or \emph{volatile} (elongated
    shape) --- a critical distinction for investment risk assessment
    \cite{munzner2014}. Hovering on Manhattan's violin reveals: min \$100,
    Q1 \$110, median \$125, Q3 \$142, max \$150.}
  \label{fig:investor_violin_borough}
\end{figure}

\subsection{Price Distribution by Room Type: Violin + Box Plot}

\begin{figure}[H]
  \centering
  \includegraphics[width=\textwidth]{figures/investor_violin_room.jpg}
  \caption{Price Distribution by Room Type --- Violin + Box composite.
    Three violins compare the \texttt{price} distribution for
    \emph{Private room} (green), \emph{Entire home/apt} (blue), and
    \emph{Shared room} (orange). Private room has the lowest median
    ($\approx$\$117) with a compact, symmetric shape, indicating a stable,
    affordable segment. Entire home/apt shows a wider interquartile range
    (Q1 $\approx$\$115, Q3 $\approx$\$145) reflecting the premium associated
    with whole-unit rentals. Shared room, counterintuitively, exhibits the
    widest upper tail (reaching $\approx$\$165) despite its theoretical
    low-cost positioning. Entire home/apt commands a price premium, but
    Private room offers a more predictable, stable market with lower
    risk volatility \cite{zervas2017}.}
  \label{fig:investor_violin_room}
\end{figure}

\subsection{Investment Yield by Borough}

\begin{figure}[H]
  \centering
  \includegraphics[width=0.75\textwidth]{figures/investor_yield.jpg}
  \caption{Investment Yield by Borough (horizontal bar chart with sequential
    diverging orange colour scale, \texttt{px.bar}). Yield (\%) is computed
    as estimated annual revenue divided by a normalised property cost proxy.
    Bars are sorted by descending yield. \textbf{Staten Island} achieves the
    highest yield at 38.9\%, followed by \textbf{Bronx} (32.1\%) and
    \textbf{Queens} (26.2\%). \textbf{Brooklyn} (18.2\%) and
    \textbf{Manhattan} (17.0\%) yield significantly less despite their higher
    absolute revenue figures, because their property acquisition costs are
    proportionally higher \cite{wachsmuth2018}. This chart directly overturns
    a naive assumption: the boroughs with the highest revenue are not the
    highest-yield investment targets.}
  \label{fig:investor_yield}
\end{figure}

\subsection{ROI Analysis: Revenue vs.\ Potential by Room Type}

\begin{figure}[H]
  \centering
  \includegraphics[width=\textwidth]{figures/investor_roi_bar.jpg}
  \caption{ROI Analysis: Actual Revenue vs.\ Full Potential Revenue by Room
    Type (grouped bar chart, \texttt{px.bar} with barmode ``group'').
    For each room type, two bars are shown: \textbf{green} = average estimated
    actual annual revenue (based on observed availability); \textbf{blue} =
    theoretical maximum revenue if listed 365 days per year
    (\texttt{price} $\times$ 365). The gap between green and blue represents
    \textbf{unrealised revenue potential}. Entire home/apt shows actual
    revenue of \$6,912--\$17,640 per borough vs.\ potential of
    \$43,800--\$45,625, implying a large efficiency gap driven by low
    \texttt{availability\_365} values ($\approx$70 days/year on average).
    Private room achieves better efficiency: \$23,666--\$39,238 actual
    vs.\ \$40,150--\$43,975 potential \cite{zervas2017}.}
  \label{fig:investor_roi}
\end{figure}

\subsection{Market Segment Statistics and Top Professional Hosts Tables}

\begin{figure}[H]
  \centering
  \includegraphics[width=\textwidth]{figures/investor_tables.jpg}
  \caption{Market Segment Statistics table (top) and Top Professional Hosts
    table (bottom). \textbf{Market Segment Statistics} provides per-room-type
    aggregates: count, average price, median price, standard deviation,
    average revenue, and ROI score. Entire home/apt (8,164 listings, avg
    \$128, ROI 17.5\%) has the highest listing count but the lowest ROI.
    Private room (3,620 listings, avg \$119, ROI 22.5\%) and Shared room
    (85 listings, avg \$117, ROI 26.6\%) achieve higher ROI due to greater
    availability utilisation. \textbf{Top Professional Hosts} ranks by
    estimated annual revenue: \emph{Corporate Housing} leads with 62 listings,
    329 reviews, and \$1,643,800 estimated revenue at \$133 average price
    \cite{wachsmuth2018}.}
  \label{fig:investor_tables}
\end{figure}

\subsection{Market Saturation Scatter Plot}

\begin{figure}[H]
  \centering
  \includegraphics[width=0.75\textwidth]{figures/investor_saturation.jpg}
  \caption{Market Saturation Analysis (scatter plot, \texttt{px.scatter}).
    The x-axis encodes the number of listings per borough (supply); the
    y-axis encodes average availability in days per year (demand pressure:
    lower availability = higher occupancy = stronger demand). A horizontal
    dashed reference line marks the average availability threshold (135 days).
    \textbf{Manhattan} and \textbf{Brooklyn} appear in the bottom-right
    quadrant (high supply, low availability $\approx$90 days), indicating
    \emph{saturated markets}. \textbf{Staten Island} appears in the top-left
    quadrant (low supply, high availability $\approx$205 days), signalling
    an \emph{undersupplied, low-demand} market. \textbf{Bronx} (top-left)
    represents an \emph{emerging opportunity} zone. This quadrant analysis
    operationalises David's market-entry decision \cite{munzner2014}.}
  \label{fig:investor_saturation}
\end{figure}

\subsection{Market Distribution by Host Category}

\begin{figure}[H]
  \centering
  \includegraphics[width=0.75\textwidth]{figures/investor_host_bar.jpg}
  \caption{Market Distribution by Host Category (vertical bar chart,
    \texttt{px.bar}). Hosts are segmented into four categories by portfolio
    size: \emph{Single (1 listing)}: blue, 8,698 listings (73.3\%);
    \emph{Small (2--5)}: green, 2,368 listings; \emph{Mega (10+)}: red,
    522 listings; \emph{Medium (6--10)}: orange, 281 listings. The
    overwhelming dominance of single-property hosts (73.3\%) confirms that
    the NYC Airbnb market remains primarily composed of individual hosts
    despite the presence of professional operators \cite{wachsmuth2018}.
    A bar chart is used here in preference to a pie chart because the
    absolute count difference between categories (8,698 vs.\ 281) is more
    accurately judged from bar height than from wedge area
    \cite{cleveland1984}.}
  \label{fig:investor_host_bar}
\end{figure}

%% ═══════════════════════════════════════════════════════════════════════════════
%%  6. DESIGN RATIONALE AND VISUAL ENCODING
%% ═══════════════════════════════════════════════════════════════════════════════
\section{Design Rationale and Visual Encoding Decisions}

\subsection{Colour Encoding Strategy}

A consistent colour-assignment policy is maintained across all views to reduce
the cognitive load of relearning encodings between perspectives \cite{ware2004}.
Borough colours are fixed: Manhattan = orange, Brooklyn = red/blue,
Queens = green, Bronx = red, Staten Island = purple. Room-type colours are
fixed: Entire home/apt = blue, Private room = green, Shared room = orange.
Sequential scales (green to red for price, blue-to-red for revenue) are used
for ordered quantitative channels \cite{moreland2009}. Diverging scales are
avoided where the data has no natural midpoint.

\subsection{Bar Charts over Pie Charts}

Following the principle that \emph{length} is a more accurate preattentive
channel than \emph{angle} or \emph{area} for magnitude judgment
\cite{cleveland1984,munzner2014}, all room-type and host-category distribution
charts use vertical or horizontal bar charts rather than pie charts. This is
particularly important when one category dominates (e.g., Entire home/apt at
68.8\%), where a pie chart would make the minor category (Shared room, 0.7\%)
nearly invisible as a thin sliver \cite{tufte1983}, while a bar chart's
y-axis scaling makes its exact count (85) readable via the annotation.

\subsection{Violin + Box over Plain Bar for Distribution}

The Investor view uses violin plots (with embedded box plots) rather than
grouped bar charts for price distribution because: (i)\ a single mean or
median bar conceals the shape of the distribution, which is the investor's
primary concern for risk assessment \cite{munzner2014}; (ii)\ the violin
width directly encodes density, allowing the user to see whether a market
is uni- or bimodal; and (iii)\ the embedded box plot retains the familiar
five-number summary \cite{tufte1983} so that users unfamiliar with violin
plots are not disoriented.

\subsection{Symbol Map for Spatial Discovery Tasks}

All geographical views use Scatter Mapbox (symbol map) rather than Choropleth
or Density Heatmap because: (i)\ the traveller and investor tasks require
locating \emph{specific listings}, not just regional patterns; (ii)\ the
dual-channel encoding (colour + size on the same mark) is only possible on
individual points \cite{ware2004}; and (iii)\ the Mapbox tile layer provides
spatial context (streets, landmarks) essential for users unfamiliar with NYC
geography.

\subsection{Filter Panel as Linked Interaction}

The shared filter panel implements a \emph{filter-and-coordinate} interaction
pattern \cite{heer2012}. Rather than brushing within a single chart, the user
sets globally-scoped filter conditions that propagate to all charts
simultaneously. The active filter state is made explicit via the ``Active:''
token strip at the top of the panel (Figure~\ref{fig:filter_panel}),
satisfying the Milestone 4 requirement that ``state must be obvious.''

%% ═══════════════════════════════════════════════════════════════════════════════
%%  7. ACCESSIBILITY
%% ═══════════════════════════════════════════════════════════════════════════════
\section{Accessibility and Inclusive Design}

\subsection{Grayscale Mode}

Approximately 8\% of males and 0.5\% of females have some form of colour
vision deficiency, with red--green colour blindness being the most prevalent
\cite{wcag21}. The Grayscale Mode (Figure~\ref{fig:theme_grayscale}) converts
all chart colour channels to luminance-ranked grayscale equivalents
\cite{moreland2009}, ensuring that information encoded solely in hue is not
lost. Because the dashboard uses redundant encoding (colour + length for bar
charts; colour + size for maps \cite{ware2004}), no information is lost in
the grayscale rendering.

\subsection{Additional Accessibility Features}

\begin{itemize}[leftmargin=1.5em]
  \item \textbf{Keyboard navigation:} All filter controls are navigable
        via Tab key.
  \item \textbf{Screen reader support:} \texttt{AccessibleName} and
        \texttt{AccessibleDescription} properties are defined for all
        interactive PySide6 widgets \cite{pyside6}.
  \item \textbf{Descriptive chart titles:} All Plotly figures carry a
        concise, informative title that summarises the chart's purpose.
  \item \textbf{Contrast:} Dark mode maintains WCAG AA contrast ratio
        \cite{wcag21} for all text elements. Light mode uses dark text
        on white.
  \item \textbf{Details-on-demand:} All geographic and statistical charts
        surface full data records on hover, making information accessible
        without relying on colour alone \cite{heer2012}.
\end{itemize}

%% ═══════════════════════════════════════════════════════════════════════════════
%%  8. KEY FINDINGS
%% ═══════════════════════════════════════════════════════════════════════════════
\section{Key Findings}

\subsection{Traveler Findings}
\begin{enumerate}[leftmargin=1.5em]
  \item Within the \$100--\$150 budget range with $\geq$100 reviews,
        Brooklyn and Queens together offer 472 qualifying listings,
        concentrated in Williamsburg, Astoria, and Flushing.
  \item The median and mean price both converge at exactly \$125/night
        under these filters, indicating a symmetric price distribution
        without extreme outlier distortion.
  \item Brooklyn achieves the highest value score ($\approx$0.22
        reviews/\$), outperforming all boroughs as a price-normalised
        popularity leader.
  \item Entire home/apt listings constitute 68.8\% of the total market,
        confirming high supply availability for the preferred room type.
\end{enumerate}

\subsection{Investor Findings}
\begin{enumerate}[leftmargin=1.5em]
  \item \textbf{Staten Island (38.9\%) and Bronx (32.1\%)} offer the
        highest investment yield despite lower absolute revenues, because
        property acquisition costs are proportionally lower
        \cite{wachsmuth2018}.
  \item The gap between actual revenue and full-potential revenue is
        largest for \emph{Entire home/apt} listings, listed on average
        for only $\approx$70 days/year. Increasing availability is the
        single most impactful lever for revenue optimisation.
  \item Private room shows better revenue efficiency and a lower-risk
        price distribution (compact violin shape).
  \item 73.3\% of hosts operate a single property, confirming that the
        market is not yet dominated by institutional investors in the
        2019 snapshot \cite{zervas2017}.
\end{enumerate}

%% ═══════════════════════════════════════════════════════════════════════════════
%%  9. CONCLUSION
%% ═══════════════════════════════════════════════════════════════════════════════
\section{Conclusion}

This project has demonstrated that a persona-driven, multi-view dashboard
can make a complex 48,895-row urban dataset analytically tractable for five
distinct user types without requiring any coding expertise from the end user.
The design decisions documented in this report --- bar charts over pie charts
for distributional comparisons \cite{cleveland1984}, violin plots over bar
charts for investor risk analysis \cite{munzner2014}, symbol maps over
choropleths for spatial discovery tasks \cite{ware2004}, and a globally-scoped
filter panel for coordinated multi-view exploration \cite{heer2012} --- are
grounded in perceptual principles and directly mapped to the task abstractions
established in Milestones 1--3 \cite{brehmer2013}.

The three-theme system (Dark, Light, Grayscale) ensures the dashboard is
accessible across environmental conditions and user capabilities \cite{wcag21},
satisfying the Milestone 4 accessibility requirement. The implemented system
responds to filter changes in under 200\,ms for datasets up to the full
47,924-row scope, meeting the responsiveness target.

Future work could extend the dashboard to the 2023--2024 Airbnb Open Data
release to analyse post-pandemic market recovery, add a temporal animation
layer to the trend charts, and incorporate neighbourhood-level median income
data to contextualise the regulatory findings \cite{wachsmuth2018}.

%% ═══════════════════════════════════════════════════════════════════════════════
%%  REFERENCES
%% ═══════════════════════════════════════════════════════════════════════════════
\bibliographystyle{plain}
\bibliography{references}

\end{document}
